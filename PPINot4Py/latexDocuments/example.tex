% !TEX TS-program = pdflatex
% !TEX encoding = UTF-8 Unicode

% This is a simple template for a LaTeX document using the "article" class.
% See "book", "report", "letter" for other types of document.

\documentclass[11pt]{article} % use larger type; default would be 10pt

\usepackage[utf8]{inputenc} % set input encoding (not needed with XeLaTeX)

%%% Examples of Article customizations
% These packages are optional, depending whether you want the features they provide.
% See the LaTeX Companion or other references for full information.

%%% PAGE DIMENSIONS
\usepackage{geometry} % to change the page dimensions
\geometry{a4paper} % or letterpaper (US) or a5paper or....
% \geometry{margin=2in} % for example, change the margins to 2 inches all round
% \geometry{landscape} % set up the page for landscape
%   read geometry.pdf for detailed page layout information

\usepackage{graphicx} % support the \includegraphics command and options

% \usepackage[parfill]{parskip} % Activate to begin paragraphs with an empty line rather than an indent

%%% PACKAGES
\usepackage{booktabs} % for much better looking tables
\usepackage{array} % for better arrays (eg matrices) in maths
\usepackage{paralist} % very flexible & customisable lists (eg. enumerate/itemize, etc.)
\usepackage{verbatim} % adds environment for commenting out blocks of text & for better verbatim
\usepackage{subfig} % make it possible to include more than one captioned figure/table in a single float
% These packages are all incorporated in the memoir class to one degree or another...

\usepackage{listings}
\usepackage{xcolor}
 
\definecolor{codegreen}{rgb}{0,0.6,0}
\definecolor{codegray}{rgb}{0.5,0.5,0.5}
\definecolor{codepurple}{rgb}{0.58,0,0.82}
\definecolor{backcolour}{rgb}{0.95,0.95,0.92}

\lstset{showstringspaces=false}
\lstdefinestyle{Python}{
    language        = Python,
    basicstyle      = \ttfamily,
    numbers=left,
    stepnumber=1,
    keywordstyle    = \color{blue},
    stringstyle     = \color{red},
    commentstyle    = \color{red}\ttfamily
}

%%% HEADERS & FOOTERS
\usepackage{fancyhdr} % This should be set AFTER setting up the page geometry
\pagestyle{fancy} % options: empty , plain , fancy
\renewcommand{\headrulewidth}{0pt} % customise the layout...
\lhead{}\chead{}\rhead{}
\lfoot{}\cfoot{\thepage}\rfoot{}

%%% SECTION TITLE APPEARANCE
\usepackage{sectsty}
\allsectionsfont{\sffamily\mdseries\upshape} % (See the fntguide.pdf for font help)
% (This matches ConTeXt defaults)

%%% ToC (table of contents) APPEARANCE
\usepackage[nottoc,notlof,notlot]{tocbibind} % Put the bibliography in the ToC
\usepackage[titles,subfigure]{tocloft} % Alter the style of the Table of Contents
\renewcommand{\cftsecfont}{\rmfamily\mdseries\upshape}
\renewcommand{\cftsecpagefont}{\rmfamily\mdseries\upshape} % No bold!

%%% END Article customizations

%%% The "real" document content comes below...

\title{PPINot4Py description}
\author{Alejandro Gómez Caballero}
%\date{} % Activate to display a given date or no date (if empty),
         % otherwise the current date is printed 

\begin{document}
\maketitle

\section{Measures}

Defining a measure is the way to indicate wich kind of operation we want to do in our dataframe, we have several tipes:


\subsection{CountMeasure}
Most basic measure, used to count the times a condition occurs.\\
It can receive as parameter a String condition o a TimeInstantCondition

\begin{lstlisting}[style = Python]
countStateTimeA = 
    DataObjectState('lifecycle_transition == "In Progress"')
countConditionTimeA = 
    TimeInstantCondition(countStateTimeA)
countMeasureTimeA = 
    CountMeasure(countConditionTimeA)

#----------------------------'OR'---------------------------

countMeasureTimeA = 
    CountMeasure('lifecycle_transition == "In Progress"')    

\end{lstlisting}

\subsection{DataMeasure}
With this measure we can obtain the first or last appareance of a value when we apply a TimeInstantCondittion

\subsection{TimeMeasure}

In this measure, we want to count how much time has happened between 1 event and other, in a linear form or in a cyclic form\\

If we call "A" to the first condition and "B" to the second, we want the appareances of the pairs A$\,\to\,$B\\
We have the following parameters:\\
\indent
FromCondition: The first TimeInstantCondition \\
\indent
ToCondition: The second TimeInstantCondition \\
\indent
TimeMeasureType: Tipe of calculation, Linear or Cyclic.\\
\indent
SingleInstanceAggFunction: Tipe of operation we want to apply.\\
\indent
Precondition: In case we want to apply a previous filter to our dataframe\\ 
\indent
FirstTo: Only works for the linear type, is True if you want the first "B" and False if you want the last "B"\\
\subsubsection{Linear type}
In this case, we are counting the first appareance of A condition and the first/last appareance of B condition, then we calculate the time difference between B and A\\
\subsubsection{Cyclic type}
In cyclic case, we count each apparition of A$\,\to\,$B apparitions and calculate an operation to this apparitions.
\subsubsection{AggFuntions}
We have 5 possible operations by defect in the program:\\

SUM: The sum of the time values of all A$\,\to\,$B apparitions\\
\indent
MAX: Max time value between the A$\,\to\,$B pairs\\
\indent
MIN: Min time value between the A$\,\to\,$B pairs\\
\indent
AVG: The average of the time values of all A$\,\to\,$B apparitions\\
\indent
GROUPBY: Raw grouped dataframe with no operation applied\\

\subsection{aggregatedMeasure}


\end{document}
